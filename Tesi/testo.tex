\section{M.I.A e I.I.A}\label{m.i.a-e-i.i.a}

I modelli di machine learning possono presentare dei rischi per la
privacy nel caso in cui facciano trasparire delle informazioni sensibili
sui dati di training attraverso il loro output. Si parla di Membership
Inference Attack se è possibile con una certa accuratezza determinare se
un certo dato fosse presente nei dati di training di un modello. Si
parla invece di Identity Inference Attack se è possibile determinare se
nel training dataset fosse presente qualche dato corrispondente ad una
certa persona avendo a disposizione un altro dato della stessa persona
(ad esempio a partire da una foto di un individuo determinare se nel
training dataset fosse presente un'altra foto dello stesso individuo).
\# Privacy nei modelli di machine learning Dobbiamo definire cosa si
intende per privacy nel machine learning: Una definizione possibile è
quella che viene chiamata ``differential privacy'' ovvero si richiede
che un attaccante non possa usare il modello per risalire a informazioni
sul training dataset che non avrebbe potuto dedurre da un altro modello
addestrato sulla stessa distribuzione del dataset. Questo quindi non
significa che non si possa dedurre alcuna informazione sulla
distribuzione dei dati di training. Infatti sarebbe una richiesta troppo
forte in quanto un modello per essere utile deve generalizzare il più
possibile le informazioni apprese dal training dataset a tutta la
distribuzione che ha generato il dataset. Richiedere che dal modello non
si possa dedurre alcuna informazione sulla distribuzione del dataset
equivale a chiedere che il modello non funzioni. Differential privacy
quindi è la richiesta che dal modello non si possa risalire a
informazioni specifiche dei dati della distribuzione che erano presenti
nel training dataset. Ad esempio un modello che identifica che una certa
categoria di persone è predisposta ad una certa malattia non viola la
differential privacy se: 1. È possibile costruire un altro modello tale
che venendo addestrato sulla stessa distribuzione troverà la stessa
relazione tra categoria di persone e malattia. 2. Non è possibile
ricavare informazioni sulle persone presenti nel training dataset,
quindi il modello deve essere resistente a M.I.A. e I.I.A. In questo
esempio si nota che violare la differential privacy è particolarmente
grave in quanto permette di accedere allo stato di salute di una
specifica persona se presente nel dataset di addestramento.

{[}{[}Comprehensive\_Privacy\_Analysis\_of\_Deep\_Learning\_Passive\_and\_Active\_White-box\_Inference\_Attacks\_against\_Centralized\_and\_Federated\_Learning.pdf{]}{]}
(qua evidenziato in rosa e anche nell'altro paper di Reza) TODO da
mettere a posto

\section{Black Box e White Box}\label{black-box-e-white-box}

Per gli inference attacks possiamo usare due paradigmi diversi. * White
box (scatola bianca) ovvero l'attaccante conosce informazioni sul
modello da attaccare come, tipo di modello utilizzato, numero di layer,
parametri. (il paper di Zhang sfrutta federated learning per ottenere
dei parametri e' quasi white box TODO) * Black box (scatola nera) ovvero
l'attaccante non ha alcuna informazione sul funzionamento interno del
modello e può solo utilizzarlo osservando gli output corrispondenti agli
input che gli fornisce. Il modello black box può essere visto come una
API a cui è possibile fare delle richieste con degli input scelti
dall'attaccante e ottenere le risposte da utilizzare per l'attacco. Ci
concentreremo su questo secondo modello in quanto è più simile ad un
caso reale in cui un attaccante effettui un M.I.A. su un modello a cui
non ha accesso direttamente. \# M.I.A. su classificatori Chiamiamo
\(\mathcal T\) il modello target su cui vogliamo svolgere l'attacco.\\
Chiamiamo \(\mathcal D_{train,\mathcal T}\) il dataset di training di
\(\mathcal T\).

Nei modelli di machine learning di classificazione, il modello determina
a quale tra \(k\) classi è più probabile appartenga l'input. Il
classificatore da in output un vettore lungo k dove ogni componente
rappresenta la probabilità che l'input appartenga alla corrispondente
classe. Ad esempio
\(\begin{bmatrix}cane\\gatto\\orso\\volpe \end{bmatrix}=\begin{bmatrix}0.6\\0.1\\0.1\\0.2 \end{bmatrix}\)
L'intuizione su cui ci basiamo è che il modello classificherà in maniera
diversa input che erano già presenti nei dati di training
(\(\mathcal D_{train,\mathcal T}\)), per esempio con una confidenza
maggiore. Ad esempio:
\(\begin{bmatrix}cane\\gatto\\orso\\volpe \end{bmatrix}=\begin{bmatrix}0.9\\0.025\\0.025\\0.05 \end{bmatrix}\)
Come mostra il Paper di REZA(aggiungi link TODO) l'overfitting del
modello da attaccare \(\mathcal T\) rende maggiori le differenze nella
confidenza della classificazione tra dati nuovi e dati già visti dal
modello durante il training. Quindi l'overfitting facilita attacchi di
inferenza sul modello.

Possiamo quindi addestrare un nuovo modello di machine learning
\(M_{inference}\) (un classificatore binario) che a partire da queste
differenze nell'output tra dati in \(\mathcal D_{train,\mathcal T}\) e
in \(\overline{ \mathcal D_{train,\mathcal T}}\) (il complemento)
determini se l'input \(\in \mathcal D_{train,\mathcal T}\) o no.

Per poter addestrare \(\mathcal M_{inference}\) avremmo bisogno dei
vettori di predizione
(e.g.~\(\begin{bmatrix}cane\\gatto\\orso\\volpe \end{bmatrix}\)) con la
corrispondente label \texttt{in} o \texttt{out} in base all'appartenenza
a \(\mathcal D_{train,\mathcal T}\).

Non abbiamo a disposizione questi dati per il modello \(\mathcal T\)
quindi creiamo una serie di ``shadow models'' \(\mathcal S_i\) che
andranno a imitare \(\mathcal T\). Siccome questi \(\mathcal S_i\) sono
creati da noi possiamo controllarne il training set
\(\mathcal D_{train,\mathcal S_i}\)({[}{[}\#creazione dataset
shadow{]}{]}) e quindi abbiamo a disposizione dei vettori di predizione
con la corrispondente label \texttt{in} e \texttt{out}. Possiamo quindi
addestrare il classificatore binario \(\mathcal M_{inference}\) in modo
che capisca se un certo input appartenga a
\(\mathcal D_{train,\mathcal S_i}\) oppure no in base al vettore di
predizione corrispondente. L'idea è che se gli shadow models
\(\mathcal S_i\) si comportano in maniera abbastanza simile a
\(\mathcal T\) la capacità di \(\mathcal M_{inference}\) di discriminare
in e out di \(\mathcal D_{train,\mathcal S_i}\) si tradurrà nella
capacità di discriminare in e out in \(\mathcal D_{train,\mathcal T}\).
In questo caso avere informazioni aggiuntive su \(\mathcal T\) (ad
esempio il tipo del modello) permette di creare dei \(\mathcal S_i\) più
simili aumentando l'accuratezza del nostro attacco. Per l'addestramento
dei \(\mathcal S_i\) avremo che \(\mathcal D_{train,\mathcal S_i}\)
contiene tutti dati con una determinata classe \(i\) (ad esempio
\(\mathcal D_{train,\mathcal S_i}\) ha tutti
cani,\(\mathcal D_{train,\mathcal S_2}\) ha tutti gatti ecc\ldots),
questo perché la distribuzione del vettore di predizione può dipendere
dalla classe dell'input (ad esempio se è più facile per un modello
classificare certi animali rispetto ad altri). Per il modello
\(\mathcal M_{inference}\) si può utilizzare qualsiasi modello che
permetta la classificazione binaria.

\begin{longtable}[]{@{}
  >{\raggedright\arraybackslash}p{(\columnwidth - 0\tabcolsep) * \real{0.0556}}@{}}
\toprule\noalign{}
\begin{minipage}[b]{\linewidth}\raggedright
\# creazione dataset shadow come creiamo i dataset shadow TODO se serve
Guardare da tutti i paper per le varie spiegazioni specialmente Reza e
GANMIA.
\end{minipage} \\
\midrule\noalign{}
\endhead
\bottomrule\noalign{}
\endlastfoot
\# GAN Le generative neural network sono una classe di modelli di deep
learning utilizzati per generare dei dati dalla stessa distribuzione dei
dati di training. Una GAN è composta da due reti neurali: una rete
generatrice che prende in input rumore e genera dei dati e una rete
discriminatrice che prende in input i dati provenienti dalla
distribuzione originale oppure quelli generati dalla rete generatrice e
cerca di determinare se siano dati originali o generati. La rete
discriminatrice viene addestrata in modo da raggiungere la massima
probabilità che riesca a distinguere dati originali da quelli generati.
La rete generatrice viene addestrata in modo da massimizzare la
probabilità che la rete discriminatrice classifichi i dati generati come
dati originali. Le due reti vengono addestrate alternandosi cercando di
tenere la rete discriminatrice vicino all'ottimalità in modo da forzare
la rete generatrice a generare dati più simili a quelli originali. \#
M.I.A su GAN HAyes ecc TODO \\
\end{longtable}

\section{M.I.A su fingerprint GAN}\label{m.i.a-su-fingerprint-gan}

\begin{center}\rule{0.5\linewidth}{0.5pt}\end{center}
